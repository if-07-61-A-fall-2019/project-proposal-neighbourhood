\documentclass[12pt]{article}
\usepackage{geometry}                % See geometry.pdf to learn the layout options. There are lots.
\geometry{letterpaper}                   % ... or a4paper or a5paper or ... 
\usepackage{graphicx}
\usepackage{amssymb}
\usepackage{amsthm}
\usepackage{epstopdf}
\usepackage[utf8]{inputenc}
\usepackage[usenames,dvipsnames]{color}
\usepackage[table]{xcolor}
\usepackage{hyperref}
\usepackage{parskip}
\DeclareGraphicsRule{.tif}{png}{.png}{`convert #1 `dirname #1`/`basename #1 .tif`.png}

\theoremstyle{definition}
\newtheorem{example}{Example}
\newenvironment{summarize}{%
   \setlength{\parindent}{0pt}
   \itshape
   \color{black}
}{}

\newenvironment{explanation}{%
   \setlength{\parindent}{0pt}
   \itshape
   \color{blue}
}{}

\newcommand{\projectname}{My Informatics World - Deep Space Simulation}
\newcommand{\projectleader}{K. Mohammed}
\newcommand{\documentstatus}{In process}
%\newcommand{\documentstatus}{Submitted}
%\newcommand{\documentstatus}{Released}
\newcommand{\version}{V. 1.6}

\begin{document}
\begin{titlepage}
\begin{flushright}
\includegraphics[scale=.5]{htlleondinglogo.png}\\
\end{flushright}

\vspace{10em}

\begin{center}
{\Huge Project Proposal} \\[3em]
\includegraphics[scale=.5]{myw.PNG}\\
\end{center}
\hfill \break
\begin{flushleft}
\begin{tabular}{|l|l|}
\hline
Project Name & \projectname \\ \hline
Project Leader & \projectleader \\ \hline
Document state & \documentstatus \\ \hline
Version & \version \\ \hline
\end{tabular}
\end{flushleft}

\end{titlepage}
\section*{Revisions}
\begin{tabular}{|l|l|l|}
\hline
\cellcolor[gray]{0.5}\textcolor{white}{Date} & \cellcolor[gray]{0.5}\textcolor{white}{Author} & \cellcolor[gray]{0.5}\textcolor{white}{Change} \\ \hline
October 14, 2019&K. Mohammed/L. Bochis/A. Doenmez&First version \\ \hline
October 18, 2019&K. Mohammed/L. Bochis/A. Doenmez&Second version \\ \hline
October 21, 2019&K. Mohammed/L. Bochis/A. Doenmez&Third version \\ \hline
October 22, 2019&K. Mohammed/L. Bochis/A. Doenmez&Fourth version \\ \hline
October 24, 2019&K. Mohammed/L. Bochis/A. Doenmez&Fifth version \\ \hline
October 25, 2019&K. Mohammed/L. Bochis/A. Doenmez&Sixth version \\ \hline
\end{tabular}
\pagebreak

\tableofcontents
\pagebreak

\section{Introduction}
Many people are negatively biased about software developers. The idea that programmers spend their time in their dark isolated room with plenty of energy drinks, having no social abilities and contacts is simply not true. We want to change this stigma.

This project should show people, that being a programmer can be pretty interesting and that it requires similar skills like in any other job. As a programmer, you always stay in touch with other colleagues and people. That is the reason why soft skills are very important in this field of work.

To change the stigma about programmers we want to create a Deep Space simulation and publish it in the Ars Electronica Center. By publishing the project, many people will get in touch with the subject "Computer Science" and they will actually see how cool it can be to be a computer scientist. The Project will show them, what a group of software developers are capable to create and which great opportunities you have in a job in software development.


\pagebreak

\section{Initial Situation}
\textbf{Explanation of Deep Space}\newline
The Ars Electronica Center is a technical museum in Linz. It has a room, that is called "deep space", which has a 16x9 meter wall projection, a floor projection of the same size and laser tracking. The deep space is capable of illustrating complex 2- and 3-dimensional simulations in 8k resolution. Guides have control over the simulation and lead the viewers through the simulation.

A lower grade Student, that is interested in computer science and technology, might find it very hard to get information and a basic overview of that field, because you need an existing knowledge to find information about computer science. 

To abolish this problem there will be an event called "My Informatics World" that will likely be organized by the HTL Leonding, the FH Hagenberg, the CoderDojo Linz, the HTL Perg, the HTL Grieskirchen, the JKU, Company Partners and the Ars Electronica Center. The goals of "My Informatics World" are \textbf{"represent the predominant communicative and creative parts of the computer science professions"}, \textbf{"show the manifold possibilities in the Upper Austrian central area to learn computer science and to earn a living with a profession in computer science"} and \textbf{"To make the teaching of computer science at primary and lower secondary level attractive to teachers and instructors and to show them ways of supporting them in this process."}.

The event will take place in the the Ars Electronica Center. Interested students will be able go into the deep space, that is located in the Ars Electronica Center, and get a brief overview about software engineering and how computer works.

\pagebreak

\section{General Conditions and Constraints}

The proposed system has to the deal with the following constraints:
\begin{itemize}
\item The model should look smooth and not complicated. 
\item The simulation should be fully controllable by the guides and only by the guides.
\item The simulation should be comprehensible for every age group, especially young students and their parents.
\item The simulation should be objective towards information and should not favor a certain school, although it was created by students of the HTL Leonding.
\item Perspective change through guide control (bird-eye view, first and third person).
\end{itemize}
\pagebreak

\section{Project Objectives and System Concepts}

\textbf{System Concepts:}
\begin{itemize}
\item{\textbf{Messenger Service}}\newline
How does a messenger work?\newline
How is the network structured?\newline
What happens in the back-end of a messenger?

This concept should explain how the communication between two or more persons works. It will visualize, how a message package is structured and how a message package is sent from one person to another with a message server in between.
\item{\textbf{Online Video Game}}\newline
What happens on the server of a multiplayer game? \newline
How is the network structured?

People often play video games without knowing what the computer does while they play. This simulation should give them a brief overview of the components that work together during a game. The simulation should visualize, how a connection to server is established and how gamer can play a multiplayer game on a server.
\item\textbf{Inside a Computer}\newline
What is inside a computer and how does it's components work? \newline
What components are in a CPU?\newline
How does a CPU work and how does it run programs?\newline
What is Assembler?\newline

This concept should give a brief overview of the core technologies of the Computer. The main focus will be the central processing unit. It will explain the different parts of the CPU and explain how they interact with each other. It will explain how an abstracted form of machine code is executed and how the CPU handles it.

\pagebreak
\item\textbf{Programming Language}\newline
What is a programming language?\newline
How does a programming language work?\newline
How is a programming language translated into machine code?

This concept should explain how an advanced programming language works and what it is. It should give a few simple examples and explain how it is converted to a language that the computer understands. An advanced programming language is easier to understand for humans because it is uses complex abstract concepts that a human can understand easier than a Computer. Because of this a higher level programming language can be compiled to a language that is understandable for computer but not for humans. 

\item\textbf{How does memory work}\newline
How does the computer dispense its memory capacity?\newline
How does a HDD store data?\newline
How can data be reached?\newline
How is data structured?\newline
What is the difference between HDD and SSD?\newline

This concept should explain what data is and how data is abstracted to zeros and ones. It should explain how a computer handles data and what a computer does to store it. It should explain how the hard drive disk stores data and processes incoming data. The simulation should also show the differences of the functionalities between HDD and SSD.

\end {itemize}

\pagebreak

{\textbf{Project objectives}}\newline
The main objective of "{\projectname}" is to provide an overview of the technology branch, especially informatics, for people of different age. Upon viewing the simulation, spectators should be able to understand code realization and the history of specific software handlings. The simulation should not rely on the know-how that already exists.\newline

\textbf{\projectname} focuses on people with different ages, especially for students at the lower-school level and their parents, for that matter we have to secure these enumerations:

\begin{itemize}
    \item The template is supposed to look seamless and not hard.
    \item For each age group, especially young students and their parents, the simulation should be understandable.
\end{itemize}

\pagebreak
\section{Opportunities and Risks}

\textbf{The project has the following opportunities:}
\begin{itemize}
\item The project will show people that software engineering is something special and will help reducing the stigma of software engineers. 
\item The project gives real working experience. 
\item The project will be used.
\item The project can lead to new connections in the IT sector.   
\item Because of entering in new parts of the IT world we can acquire a lot of knowledge about technologies that are new for us.
\end{itemize} 

\textbf{The following risk have to be taken into account:}
\begin{itemize}
\item \textbf{Expectation risk:} Because the project will be used and viewed by real People it has to fulfill many expectations.
\item \textbf{Schedule risk:} The schedule will be strict because we have to comply with the management and expectations of the AEC.
\item \textbf{Performance risk:} The project will not produce results that are consistent with the project specifications.
\item \textbf{Enough Live-testing:} To see the result on the deep space, short time for live testing is one of the big risks.
\end{itemize}

\pagebreak
\section{Planning}

\textbf{Milestones:}
\newline

\begin{tabular}{|l|l|}
\hline
\cellcolor[gray]{0.5}\textcolor{white}{Due Date} & \cellcolor[gray]{0.5}\textcolor{white}{Milestone} \\ \hline
November 12, 2019&Working programming environment \\ \hline
November 26, 2019&First steps in Unity\\ \hline
January 17, 2019&First prototype of planned Simulation \\ \hline
March 10, 2020&Working Simulation \\ \hline
March 17, 2020&First live test in deep space\\ \hline
March 21, 2020&Running Simulation with good design\\ \hline
\end{tabular}


\textbf{Project roles:}
\begin{itemize}
    \item Kevin Mohammed (CEO, Project leader)
    \item Anil Doenmez (CTO, Software and Design Architect)
    \item Lukas Bochis (COO, Team member)
\end{itemize}

\textbf{Needed resources:}
\begin{itemize}
    \item Unity (Software)
    \item Access to deep space for live testing
\end{itemize}

\textbf{Answering of following questions:}
\begin{itemize}
\item \textbf{When will the project end?}
The project will end in October 2020.
\item \textbf{When will the project start?}
The project will start on 05.11.2019
\item \textbf{When will be a first prototype available?}
The first prototype will be available on January 17.
\item \textbf{When does implementation work start?} 
31.10.2019
\item \textbf{What are the big blocks of work to be done?} Understanding the the system that the AEC uses.
Learn Unity and the work space around it.
\item \textbf{Is this work doable in the given period of time?} Yes. There is enough time given so we can perfect the project. If the project will not be finished before summer break, we will also get time in the winter semester of the next year.
\item \textbf{Do we need any other stuff to make our work (licenses, servers, ...)?}\newline
Simplifies our job: Image materials for visualising components
\end{itemize}

\end{document}